\documentclass[12pt, letterpaper]{article}
\title{Android Pong Game Design Document}
\author{Michael McCulloch}
\begin{document}
\maketitle
\pagebreak
\tableofcontents
\pagebreak
\section{Game Overview}
    \subsection{Concept}
    Table-tennis like competitive game.
    \subsection{Genre}
    Classic Arcade Game.
    \subsection{Target Audience}
    Your senile grandmother.
    \subsection{Game Flow Summary}
    \begin{itemize}
        \item[\textbf{Menu}] Offers single- and multiplayer Options.
        \item[\textbf{Game}] Playing the game.
        \item[\textbf{Pause}] Option to Resume, Restart, or Quit.
    \end{itemize}
    \subsection{Look and Feel}
    Classic arcade game, complete with 16-bit sound effects.
\section{Gameplay \& Mechanics}
    \subsection{Gameplay}
        \subsubsection{Progression}
        As time elapses since the last point scored, the ball accelerates to $\infty$. Incremental acceleration occures upon each paddle-ball collision.
        \subsubsection{Objectives}
        Bounce the ball out of bounds on the opposing side to score a point
    \subsection{Mechanics}
        \subsubsection{Objects, Physics \& Movement}
        \begin{itemize}
            \item[\textbf{Ball}] Constant velocity except when accelerating on collision with paddle. Bounces across norm of collided surface.
            \item[\textbf{Paddle}] Vertical movement only. Has mass $\rightarrow$ inertia (ie. Maximum rate of acceleration).
            \item[\textbf{Walls}] Top and bottom of screen only. Has collisions.
            \item[\textbf{Score Counter}] Positioned on the top of the screen, on the left and right of the Vertical Divider for P1 and P2, respectively.
            \item[\textbf{Vertical Divider}] Aesthetic purposes only. No collisions. 
        \end{itemize}
        \subsubsection{Actions}
        Paddles follow movement of touch. Left paddle follows touch input on left side, Right paddle follows right.
    \subsection{Game Options}
\section{Interface}
    \subsection{Visual System}
        \subsubsection{Menu}
        Text enclosed in a box.
        \begin{itemize}
            \item[\textbf{Singleplayer}] Player takes P1 paddle, AI assumes control of P2. Difficulty option presented. Difficulty = Easy, Medium, Hard, Impossible.
            \item[\textbf{Multiplayer}] Competitive multiplayer. Both paddles receive input from touch on their respective sides of the screen.
        \end{itemize}
        \subsubsection{Pause Menu}
        In game, tap the option button to open the pause menu.
        \begin{itemize}
            \item[\textbf{Resume}] Unpause the game
            \item[\textbf{Restart}] Restart the game
            \item[\textbf{Quit}] Exit the game.
        \end{itemize}
        \subsubsection{Game Interface}
        Play area is entire screen. Can play in landscape/portrait, and alternate during play. Bar bisects play area along shortest edge. In landscape, score sits just under top of screen, adjacent to the dividing bar. In portrait, score lies above and below bar on right hand side. Pause menu activated by tapping pause button.
    \subsection{Control System}
    Bar bisects play area into A1 and A2. Paddle $P_i$ is follows touch input in area $A_i$. Paddle.Y always equal to Touch.Y, so the player can tap to snap (within acceleration limits) the paddle to desired position.
    \subsection{Audio, Music \& Sound}
    \begin{itemize}
        \item Ball bounces of any object.
        \item Ball out of bounds.
        \item Any button pressed.
    \end{itemize}
\section{Artificial Intelligence}
    \subsection{Opponent}
        Paddle.Y$_t$ = Ball.Y$_{t-X}$ where X is a fixed time delay for some difficulty. On easy difficulty, the time delay is large. On impossible, there is no delay.
\section{Technical}
    \subsection{Target Hardware}
    Android
\end{document}